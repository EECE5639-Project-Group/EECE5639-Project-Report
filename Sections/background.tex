Diabetic retinopathy is a progressive disease that necessitates early and accurate detection to prevent severe vision loss. With the increasing prevalence of diabetes worldwide, efficient and scalable diagnostic technologies are crucial. Computer vision has emerged as a powerful tool in this context, enabling the automated detection and classification of diabetic retinopathy.

Computer vision has become increasingly prevalent in medical diagnostics \cite{esteva_deep_2021}. Specifically for diabetic retinopathy, it typically involves the use of classifiers that categorize a single fundus image into one of five stages of the disease \cite{mateen_fundus_2019}. The VGG model, known for its robust feature extraction capabilities, is particularly favored in image recognition tasks \cite{mateen_fundus_2019, kaur_automated_2019, ha_image_2018, varshni_pneumonia_2019, chaib_deep_2017, lu_feature_2017, majib_vgg-scnet_2021, tammina_transfer_2019}. It effectively identifies key indicators of diabetic retinopathy such as microaneurysms, exudates, hemorrhages, and abnormal vascularization \cite{hwang_optical_2015-2}. Existing applications of this model have achieved accuracies exceeding 85\% in diagnosing the condition \cite{noauthor_diabetic_nodate}.

Despite these advancements, the opaque ``black box'' nature of deep learning models poses a significant barrier to their adoption in clinical settings \cite{guidotti_survey_2018}. Traditionally, the decision-making processes of these models are not transparent, leaving physicians without a clear understanding of why certain decisions were made \cite{guidotti_survey_2018}. To overcome this challenge, researchers have developed explainability techniques like Grad-CAM \cite{selvaraju_grad-cam_2017}. These techniques provide visual explanations by highlighting influential regions in the images at one of the final layers of the network. This not only helps bridge the gap between model decision-making and user interpretability but also supports physicians in integrating their clinical expertise with model suggestions for better-informed decision-making.