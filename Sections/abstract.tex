\begin{abstract}

Diabetic retinopathy (DR) is a leading cause of vision impairment among diabetic patients, with early detection crucial for effective management and treatment \cite{kollias_diabetic_2010}. This paper presents a computer vision approach to the diagnosis of diabetic retinopathy using a VGG-based convolutional neural network (CNN) \cite{simonyan_very_2015, lecun_gradient-based_1998}. Our model is trained on a diverse dataset of fundus camera images to classify DR into various stages of severity \cite{emma_dugas_diabetic_2015}. To address the critical need for transparency and trust in medical diagnostic tools, our approach incorporates the generation of saliency maps, which highlight the specific areas within the images that influence the model's predictions \cite{amann_explainability_2020,simonyan_deep_2014, selvaraju_grad-cam_nodate}. This visualization aids in demystifying the model's decision-making process, providing healthcare professionals with valuable insights into the reasoning behind the diagnoses. The saliency overlay not only enhances the interpretability of the automated system but also serves to augment the diagnostic process by focusing attention on areas of potential concern. By presenting these findings alongside the model's classification, our tool is designed to support, rather than replace, the clinical judgment of physicians. This paper demonstrates the potential of computer vision techniques to not only automate the detection of diabetic retinopathy but also to contribute meaningful insights for improved patient care. Our results affirm the efficacy and reliability of our model, promoting its integration as a supportive tool in clinical settings.

\end{abstract}
\begin{IEEEkeywords}
Computer Vision, Diabetes, Retinopathy
\end{IEEEkeywords}