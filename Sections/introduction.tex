Diabetic retinopathy is an eye disease associated with both Type 1 and Type 2 diabetes that causes damage to the blood vessels in the retina \cite{noauthor_diabetic_nodate}. Left untreated, diabetic retinopathy can cause permanent vision loss \cite{lee_epidemiology_2015}. The Center For Disease Control (CDC) estimates that there are 9.6 million people in the United States living with Diabetic Retinopathy, of whom 1.84 million have vision-threatening diabetic retinopathy \cite{noauthor_diabetic_2023}. Since diabetic retinopathy is caused by swelling and leaking in the vasculature, which is associated with elevated blood glucose over long periods of time, it is imperative that the disease be caught earlier rather than later \cite{hammes_diabetic_2011}. Notably, early-detection enables non-invasive treatments involving controlling blood glucose levels, blood pressure, and cholesterol levels \cite{stitt_progress_2016}. However, late-stage diabetic retinopathy treatments are generally more invasive, including photocoagulation, steroid injections, and vitrectomy \cite{stitt_progress_2016}. 

Diabetic retinopathy is typically divided into 5 stages: \textbf{0: No DR} - no hemorrhaging, microaneurysms, or abnormal vascularization occurs, \textbf{1: Mild Non-proliferative DR} - swelling in small blood vessels occurs, \textbf{2: Moderate Non-proliferative DR} - blood vessels become blocked, \textbf{3: Severe Non-proliferative DR} - ischemia occurs, blot hemorrhaging occurs, and abnormal vascularization may occur, and \textbf{4: Proliferative DR} - new, abnormal and fragile, blood vessels begin to grow in the eye \cite{yun_identification_2008}.

Our goal is to introduce a computer vision model that aids physicians in accurately diagnosing the various stages of diabetic retinopathy. This model aims to serve as a critical tool, enhancing the precision of clinical assessments and facilitating a more streamlined diagnostic process. By integrating visual data analysis, the model provides valuable support in clinical decision-making, offering physicians insights that are essential for monitoring disease progression over time. Ultimately, our goal is to equip healthcare providers with technology that not only improves diagnostic accuracy but also contributes to the longitudinal study of diabetic retinopathy, paving the way for better patient outcomes.