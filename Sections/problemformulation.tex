The goal of our project is to advance the diagnosis of diabetic retinopathy, a leading cause of blindness among adults worldwide, by implementing a dual-purpose computer vision algorithm. This algorithm is designed not only to improve diagnostic accuracy but also to enhance transparency in the diagnostic process. Current methods for detecting and assessing diabetic retinopathy depend heavily on the manual examination of retinal images by skilled clinicians. These methods are often limited by the availability of experts and can suffer from subjective variability in diagnosis. As the prevalence of diabetes increases globally, there is an urgent need for a more scalable and consistent approach to diagnosing this vision-threatening condition.

Our objectives are twofold. Firstly, we aim to accurately classify the severity of diabetic retinopathy across its spectrum, from mild to severe stages. The ability to classify with high precision is crucial as it directly informs treatment options and management strategies, potentially leading to better patient outcomes. Secondly, we seek to increase the accountability and trustworthiness of AI in medical diagnostics by generating saliency maps. These maps are intended to visually represent the critical features within the retinal images that influence the AI's decision-making process. By doing so, we aim to provide clinicians and patients with clear, visual explanations of the AI's diagnoses, promoting transparency and understanding.

In summary, our work seeks to address the significant challenge of scaling diabetic retinopathy diagnostics while maintaining the quality and reliability expected in healthcare. By achieving these goals, we anticipate that the confidence in AI-assisted diagnostics will be significantly bolstered, leading to wider acceptance and use of these technologies in clinical settings. This approach not only promises to enhance the capacity for early detection and treatment of diabetic retinopathy but also sets a precedent for the application of AI in other areas of medical imaging and diagnosis.